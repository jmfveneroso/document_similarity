\documentclass{article}
\usepackage{graphicx}
\usepackage[utf8]{inputenc}
\usepackage[T1]{fontenc}
\usepackage{pgfplots}
\usepackage{pgfplotstable} 
\usepackage{titlesec}
\usepackage{lipsum}
\usepackage{authblk}

\begin{document}

\title{NLP: Trabalho Prático 1 (Word2Vec)}
\author{João Mateus de Freitas Veneroso}
\affil{Departamento de Ciência da Computação da Universidade Federal de Minas Gerais}

\maketitle

\section{Introdução}

Esse relatório descreve a implementação do trabalho prático 1 da disciplina NLP. 
O trabalho consistiu em comparar o conteúdo de 70 obras clássicas
disponibilizadas publicamente pelo projeto Gutemberg com a utilização de
vetores semânticos. A versão mais recente do código pode ser obtida em: \\
https://github.com/jmfveneroso/word2vec\_text\_similarity.

\section{Decisões de Projeto}

O conteúdo de 70 obras clássicas foi o obtido no endereço: \\
https://www.gutenberg.org/ em formato de texto simples. Referências alheias ao conteúdo original das obras foram removidas 
manualmente à medida do possível para evitar contaminação dos resultados por metadados e notas
dos editores. A pontuação foi removida e as letras maiúsculas foram convertidas em minúsculas.
Por último, o algoritmo de \textit{stemming} de Porter foi aplicado para extrair os radicais das
palavras. Os dados tratados foram divididos em 70 arquivos contendo \textit{tokens} referentes 
ao conteúdo das respectivas obras separados por espaços.

A coleção completa possui 10.593.149 palavras para um vocabulário de 22.584 termos únicos (após 
a realização do \textit{stemming}). Se utilizássemos o vocabulário completo para calcular as matrizes de similaridade,
elas ficariam com aproximadamente 510.037.056 entradas por documento, resultando em um espaço de aproximadamente 2.04 GB.
Portanto, essa abordagem se torna rapidamente inviável em máquinas convencionais. A fim de tratar o problema, apenas
os 1000 termos com maior frequência foram utilizados, excluindo-se as \textit{stop words} mais comuns da
língua inglesa. 

Essa abordagem tende a subestimar a dissimilaridade entre as obras, pois ela ignora a maior parte dos
termos únicos ou pouco frequentes na coleção, no entanto, esses termos são provavelmente os identificadores mais
icônicos de um autor ou obra. Por outro lado, o enfoque nas palavras de uso corriqueiro possibilita
um cálculo mais robusto dos vetores semânticos, uma vez que elas aparecem em um número maior de contextos 
distintos na coleção.

Os vetores semânticos foram calculados com base na implementação canônica do Word2Vec. O modelo
utilizado foi o \textit{Skip Gram} com janela de tamanho 10 e vetores de 100 dimensões. Adicionalmente, 
foi realizado um teste com vetores de 200 dimensões. Em ambos os casos, uma matriz de similaridade
quadrada de tamanho 1000 X 1000 com as distâncias por cosseno dos vetores semânticos foi computada para cada
uma das obras e elas foram comparadas por meio da norma de Frobenius, que pode ser definida pela expressão:

\[
\sqrt{\sum_{i = 1}^n\sum_{j = 1}^n (a_{ij} - b_{ij})^2}
\]

A norma de Frobenius é menor quanto mais similares forem as matrizes. No caso desse trabalho, a norma tende a ser menor
quando duas obras possuem conteúdos similares ou, mais especificamente, quando os vetores semânticos são similares. 
A norma de Frobenius fica com o valor zero para conteúdos idênticos.

\section{Resultados}

As matrizes das 70 obras foram comparadas uma a uma por meio da norma de Frobenius resultando em um total de 
4.900 combinações. Os resultados completos estão disponíveis no endereço: \\
https://docs.google.com/spreadsheets/d/1DvfqgbsrPrO40C6j0EnztfBqnmAAlRcKunEkCjL2-oI/edit?usp=sharing.

\begin{table}[h]
  \begin{center}
    \tiny
    \begin{tabular}{ | l | l | l |}
    \hline
    Obra & Obra mais similar & Similaridade \\ 
    \hline
    A Study in Scarlet & The Adventures of Sherlock Holmes & 387.81 \\
    Adventures of Huckleberry Finn & Moby Dick & 450.78 \\
    Anna Karenina & The Brothers Karamazov & 202.09 \\
    Wealth of Nations & David Copperfield & 311.37 \\
    Emma & Great Expectations & 313.90 \\
    The Adventures of Sherlock Holmes & The Return of Sherlock Holmes & 268.01 \\
    Gulliver's Travels & Wuthering Heights & 321.62 \\
    Crime and Punishment & Jane Eyre - An Autobiography & 254.64 \\
    Autobiography of Benjamin Franklin & The Adventures of Sherlock Holmes & 389.67 \\
    Pride and Prejudice & Sense and Sensibility & 301.96 \\
    The Hound of the Baskervilles & Wuthering Heights & 366.76 \\
    The Adventures of Tom Sawyer & On The Duty Of Civil Disobedience & 334.46 \\
    Les Misérables & War and Peace & 165.14 \\
    Beyond Good and Evil & The Works of Edgar Allan Poe — 2 & 440.72 \\
    Sense and Sensibility & Pride and Prejudice & 301.96 \\
    The Return of Sherlock Holmes & The Adventures of Sherlock Holmes & 268.01 \\
    The Jungle Book & A Tale of Two Cities & 494.11 \\
    The Brothers Karamazov & Anna Karenina & 202.09 \\
    Il Principe & A Tale of Two Cities & 485.22 \\
    The Sign of the Four & The Hound of the Baskervilles & 397.01 \\
    Three Men in a Boat & Wuthering Heights & 343.43 \\
    The Count of Monte Cristo & Les Misérables & 170.57 \\
    Second Treatise of Government & Jane Eyre - An Autobiography & 497.94 \\
    Treasure Island & Wuthering Heights & 374.28 \\
    War and Peace & Les Misérables & 165.14 \\
    Leviathan & Ulysses & 347.68 \\
    Ulysses & David Copperfield & 252.14 \\
    The Life and Adventures of Robinson Crusoe & Moby Dick & 344.68 \\
    The Republic & Jane Eyre - An Autobiography & 294.32 \\
    Utopia & Gulliver's Travels & 443.41 \\
    A Doll's House: a play & Oliver Twist & 515.30 \\
    A Tale of Two Cities & Oliver Twist & 256.20 \\
    A Christmas Carol in Prose & Wuthering Heights & 449.12 \\
    Common Sense & Leviathan & 508.89 \\
    Frankenstein & The Works of Edgar Allan Poe — 2 & 334.76 \\
    Pygmalion & The Adventures of Sherlock Holmes & 465.84 \\
    Anne of Green Gables & Wuthering Heights & 314.64 \\
    Alice's Adventures in Wonderland & Through the Looking-Glass & 487.63 \\
    Don Quixote & The Count of Monte Cristo & 184.74 \\
    Moby Dick & Great Expectations & 257.38 \\
    The Importance of Being Earnest & Oliver Twist & 525.55 \\
    David Copperfield & Anna Karenina & 205.36 \\
    Around the World in Eighty Days & Wuthering Heights & 360.48 \\
    Paradise Lost & Moby Dick & 416.17 \\
    The Scarlet Letter & Wuthering Heights & 341.11 \\
    Great Expectations & Oliver Twist & 242.94 \\
    Candide & Frankenstein & 423.64 \\
    Songs of Innocence, and Songs of Experience & The Complete Works of Shakespeare & 396.47 \\
    The Works of Edgar Allan Poe — 1 & On The Duty Of Civil Disobedience & 300.24 \\
    Jane Eyre - An Autobiography & Great Expectations & 246.67 \\
    Dracula & Great Expectations & 246.40 \\
    The Complete Works of Shakespeare & War and Peace & 167.77 \\
    The Works of Edgar Allan Poe — 2 & On The Duty Of Civil Disobedience & 307.53 \\
    Oliver Twist & Great Expectations & 242.94 \\
    Metamorphosis & Don Quixote & 519.39 \\
    The Divine Comedy & Paradise Lost & 495.44 \\
    On The Duty Of Civil Disobedience & Wuthering Heights & 295.70 \\
    The Picture of Dorian Gray & Wuthering Heights & 334.10 \\
    Peter Pan & Wuthering Heights & 428.71 \\
    The Iliad & Jane Eyre - An Autobiography & 285.16 \\
    The Secret Adversary & Wuthering Heights & 316.07 \\
    The Mysterious Affair at Styles & The Secret Adversary & 390.30 \\
    The King James Version of the Bible & The Complete Works of Shakespeare & 182.66 \\
    Wuthering Heights & The Return of Sherlock Holmes & 291.84 \\
    The Strange Case of Dr. Jekyll and Mr Hyde & Treasure Island & 468.26 \\
    The Tragedy of Romeo and Juliet & Ulysses & 513.75 \\
    The Time Machine & Treasure Island & 486.86 \\
    The War of the Worlds & Wuthering Heights & 401.09 \\
    The Wonderful Wizard of Oz & Oliver Twist & 507.84 \\
    Through the Looking-Glass & Alice's Adventures in Wonderland & 487.63 \\
    \hline
    \end{tabular}
    \caption{Obras mais similares com vetores de 100 dimensões} 
    \label{tab:most_similar_100}
  \end{center}
\end{table}

De acordo com a tabela \ref{tab:most_similar_100} podemos observar que os dois textos mais similares 
são \textit{War and Peace} de Leo Tolstoy, publicado em 1869, e \textit{Les Miserábles} de Victor Hugo, publicado em 1862, 
com uma norma de Frobenius igual à 165,14. Ambas são obras realistas tardias do final do século XIX e
provavelmente fazem um uso similar da linguagem, pelo menos nas traduções para a língua inglesa. Também
podemos perceber que obras com o mesmo autor tendem a apresentar alta similaridade como \textit{The Brothers Karamazov}
e \textit{Anna Karenina} de Tolstoy, \textit{Alice's Adventures in Wonderland} e \textit{Through the Looking-Glass} de Lewis Carrol e 
\textit{Oliver Twist} e \textit{Great Expectations} de Charles Dickens. 

O romance \textit{Wuthering Heights} foi o único a aparecer 13 vezes como a obra mais similar entre às 70 obras 
analisadas. Essas 13 obras são: \textit{Gulliver’s Travels}, \textit{The Hound of the Baskervilles}, \textit{Three Men in a Boat},
\textit{Treasure Island}, \textit{A Christmas Carol in Prose}, \textit{Anne of Green Gables,
Around the World in Eighty Days, The Scarlet Letter, On The Duty Of Civil Disobedience,
The Picture of Dorian Gray, Peter Pan, The Secret Adversary} e \textit{The War of the Worlds}. A razão para isso
não é facilmente explicada, pois apesar de \textit{Wuthering Heights} ser uma tragédia icônica de meados do
século XIX, ela não possui muitos \textit{features} em comum com as 13 obras citadas, com exceção
talvez de \textit{The Scarlet Letter}. Uma característica em comum dessas 13 obras, no entanto, é que em sua maioria
são obras de fantasia voltadas para um público infanto-juvenil. Talvez o estilo de escrita
utilizado por Emily Brontë tenha algo em comum com essas obras, mas isso é improvável
uma vez que \textit{Wuthering Heights} é considerado um dos maiores romances da língua inglesa e também é reconhecido
pela sua complexidade. Esse fato talvez revele uma possível falha nas premissas do algoritmo ou mesmo uma
falha de implementação, no entanto, os coeficientes de similaridade para a maior parte dos outros casos
são facilmente explicáveis e são coerentes considerando o conteúdo das obras selecionadas.

Uma outra relação de similaridade interessante foi encontrada entre \textit{The King James Version of the Bible} e
\textit{The Complete Works of Shakespeare}, que possuem uma norma de Frobenius de 182,66. A razão para esse grau alto 
de similaridade provavelmente é a proximidade das data de publicação das obras: 1611 no caso de \textit{The King 
James Version of the Bible} e 1589-1613 para as obras de Shakespeare; e também o país de origem: a Inglaterra. 
Dados esses fatores, é compreensível que as obras façam uma utilização similar da língua inglesa. E, de fato,
observamos uma linguagem rebuscada significativamente diferente das outras obras presentes na coleção estudada.

Já a obra mais dissimilar entre todas as obras estudadas foi \textit{Songs of Innocence, and Songs of Experience}
de William Blake tendo sido a obra mais dissimilar em 51 das 70 comparações. Provavelmente, a razão para isso é
que a obra em questão é uma coletânea de poemas que apresenta uma estrutura significativamente diferente em 
relação às demais obras escritas em prosa. 

\begin{table}[h]
  \begin{center}
    \tiny
    \begin{tabular}{| l | l | l |}
    \hline
Índice & Obra & Autor(a) \\
    \hline
1 & A Study in Scarlet & Arthur Conan Doyle \\
2 & Adventures of Huckleberry Finn & Mark Twain \\
3 & Anna Karenina & Leo Tolstoy \\
4 & Wealth of Nations & Adam Smith \\
5 & Emma & Jane Austen \\
6 & The Adventures of Sherlock Holmes & Arthur Conan Doyle \\
7 & Gulliver's Travels & Jonathan Swift \\
8 & Crime and Punishment & Fyodor Dostoyevsky \\
9 & Autobiography of Benjamin Franklin & Benjamin Franklin \\
10 & Pride and Prejudice & Jane Austen \\
    \hline
    \end{tabular}
    \caption{Primeiras 10 obras da coleção} 
    \label{tab:books}
  \end{center}
\end{table}

\begin{table}[h]
  \begin{center}
    \tiny
    \begin{tabular}{| l | l | l | l | l | l | l | l | l | l | l |}
    \hline
 & 1 & 2 & 3 & 4 & 5 & 6 & 7 & 8 & 9 & 10 \\
    \hline
1 & 0.00 & 565.10 & 606.58 & 563.03 & 450.10 & 387.81 & 430.16 & 498.56 & 464.97 & 441.18 \\
2 & 565.10 & 0.00 & 491.42 & 504.86 & 490.99 & 484.41 & 503.35 & 464.59 & 569.54 & 521.14 \\
3 & 606.58 & 491.42 & 0.00 & 316.25 & 424.03 & 481.15 & 495.59 & 275.92 & 582.96 & 487.70 \\
4 & 563.03 & 504.86 & 316.25 & 0.00 & 414.10 & 463.75 & 443.60 & 337.29 & 538.33 & 461.32 \\
5 & 450.10 & 490.99 & 424.03 & 414.10 & 0.00 & 349.05 & 362.39 & 350.04 & 436.80 & 326.17 \\
6 & 387.81 & 484.41 & 481.15 & 463.75 & 349.05 & 0.00 & 346.89 & 379.61 & 389.67 & 349.61 \\
7 & 430.16 & 503.35 & 495.59 & 443.60 & 362.39 & 346.89 & 0.00 & 398.13 & 406.16 & 363.44 \\
8 & 498.56 & 464.59 & 275.92 & 337.29 & 350.04 & 379.61 & 398.13 & 0.00 & 486.60 & 402.21 \\
9 & 464.97 & 569.54 & 582.96 & 538.33 & 436.80 & 389.67 & 406.16 & 486.60 & 0.00 & 449.59 \\
10 & 441.18 & 521.14 & 487.70 & 461.32 & 326.17 & 349.61 & 363.44 & 402.21 & 449.59 & 0.00 \\
    \hline
    \end{tabular}
    \caption{Matriz de similaridade para as primeiras 10 obras (vetores com dimensionalidade 100)} 
    \label{tab:matrix_100}
  \end{center}
\end{table}

\begin{table}[h]
  \begin{center}
    \tiny
    \begin{tabular}{| l | l | l | l | l | l | l | l | l | l | l |}
    \hline
 & 1 & 2 & 3 & 4 & 5 & 6 & 7 & 8 & 9 & 10 \\
    \hline
1 & 0.00 & 574.38 & 615.91 & 562.38 & 449.66 & 371.01 & 421.97 & 494.98 & 458.30 & 438.20 \\
2 & 574.38 & 0.00 & 501.49 & 507.95 & 499.24 & 497.75 & 518.09 & 475.00 & 582.71 & 533.07 \\
3 & 615.91 & 501.49 & 0.00 & 317.41 & 431.67 & 499.03 & 515.04 & 281.45 & 597.52 & 503.37 \\
4 & 562.38 & 507.95 & 317.41 & 0.00 & 410.47 & 467.27 & 447.66 & 334.01 & 541.67 & 464.62 \\
5 & 449.66 & 499.24 & 431.67 & 410.47 & 0.00 & 353.55 & 367.06 & 350.18 & 437.58 & 334.57 \\
6 & 371.01 & 497.75 & 499.03 & 467.27 & 353.55 & 0.00 & 342.17 & 385.23 & 388.76 & 348.52 \\
7 & 421.97 & 518.09 & 515.04 & 447.66 & 367.06 & 342.17 & 0.00 & 407.51 & 404.75 & 361.93 \\
8 & 494.98 & 475.00 & 281.45 & 334.01 & 350.18 & 385.23 & 407.51 & 0.00 & 494.31 & 408.39 \\
9 & 458.30 & 582.71 & 597.52 & 541.67 & 437.58 & 388.76 & 404.75 & 494.31 & 0.00 & 451.02 \\
10 & 438.20 & 533.07 & 503.37 & 464.62 & 334.57 & 348.52 & 361.93 & 408.39 & 451.02 & 0.00 \\
    \hline
    \end{tabular}
    \caption{Matriz de similaridade para as primeiras 10 obras (vetores com dimensionalidade 200)} 
    \label{tab:matrix_200}
  \end{center}
\end{table}

As tabelas \ref{tab:matrix_100} e \ref{tab:matrix_200} mostram a matriz de similaridade para as 10
primeiras obras da coleção e a tabela \ref{tab:books} mostra o nome e o autor das respectivas obras.
Analisando o conteúdo das matrizes, podemos perceber que a mudança de dimensionalidade dos vetores 
não afetou significativamente as relações de similaridade. Portanto, vetores com 100 dimensões parecem
ser suficientes para modelas as interações semânticas complexas nessa pequena coleção de documentos.

\section{Conclusão}

Esse relatório descreveu a implementação do trabalho prático 1 da disciplina NLP. Os experimentos realizados
calcularam matrizes de similaridade para 70 obras clássicas com a utilização de vetores semânticos de
100 e 200 dimensões. O algoritmo mostrou ter uma boa capacidade de identificar similaridades entre as obras,
no entanto, alguns casos, como o livro Wuthering Heights, apresentaram resultados insatisfatórios. De forma
geral, os vetores de 200 dimensões não mostraram grande variação em relação aos resultados calculados com
vetores de 100 dimensões.

\end{document}
